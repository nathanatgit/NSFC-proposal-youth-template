%% Style definitions

% define cref names
\crefname{figure}{图}{图}
\crefname{table}{表}{表}
\crefname{equation}{公式}{公式}
\crefname{algorithm}{算法}{算法}
% clear footers and headers
\pagestyle{empty}

% set font family
\setCJKmainfont[Mapping=tex-text,FallBack=SimSun-ExtB]{SimSun}
\setmainfont{Times New Roman}

% Create a new font family for both English and Chinese characters (KaiTi)
\newfontfamily\kaitienglish{KaiTi}
\newCJKfontfamily\kaitichinese{KaiTi}
\newcommand{\kaiti}{%
    \kaitienglish%
    \kaitichinese%
    \CJKsetecglue{}%
}

% HeiTi for both Chinese and English (override ctex's definition)
\newfontfamily\heitienglish{SimHei}
\newCJKfontfamily\heitichinese{SimHei}
\renewcommand{\heiti}{%
    \heitienglish%
    \heitichinese%
    \CJKsetecglue{}%
}

\newcommand{\tabfont}{\fontsize{10.5}{20}\selectfont}     % default table style 五号,固定行距20pt
\captionsetup{labelsep=space}   % remove the default colon between numbering and texts, remove this line and the caption will become 表1: 表标题 or 图1: 图表题
\captionsetup{font=singlespacing,labelsep=quad}% caption seperator: space. E.g 表1 表标题 or 图1 图表题
\captionsetup[subfloat]{labelformat=simple}
\renewcommand{\captionfont}{\heiti \wuhao}
% 图序及图题居中置于图的下方,表序及表题居中置于表的上方。
\captionsetup[table]{position=top,skip=0pt}
\captionsetup[figure]{position=bottom,skip=0pt,belowskip=-10.5pt}
% 图、表等与其前后的正文之间要有一行的间距
\textfloatsep = 20pt plus 0pt minus 10.5pt
\floatsep = 10.5pt plus 0pt minus 10.5pt
\intextsep= 20pt plus 0pt minus 10.5pt

% 常用字号
\newcommand{\chuhao}{\fontsize{42pt}{42pt}\selectfont}
\newcommand{\xiaochu}{\fontsize{36pt}{36pt}\selectfont}
\newcommand{\yihao}{\fontsize{26pt}{26pt}\selectfont}
\newcommand{\xiaoyi}{\fontsize{24pt}{24pt}\selectfont}
\newcommand{\erhao}{\fontsize{22pt}{22pt}\selectfont}
\newcommand{\xiaoer}{\fontsize{18pt}{18pt}\selectfont}
\newcommand{\sanhao}{\fontsize{16pt}{22pt}\selectfont}
\newcommand{\xiaosan}{\fontsize{15pt}{22pt}\selectfont}
\newcommand{\sihao}{\fontsize{14pt}{22pt}\selectfont}
\newcommand{\xiaosi}{\fontsize{12pt}{22pt}\selectfont}
\newcommand{\wuhao}{\fontsize{10.5pt}{18pt}\selectfont}
\newcommand{\xiaowu}{\fontsize{9pt}{9pt}\selectfont}
\newcommand{\liuhao}{\fontsize{7.5pt}{7.5pt}\selectfont}
\newcommand{\xiaoliu}{\fontsize{6.5pt}{6.5pt}\selectfont}
\newcommand{\qihao}{\fontsize{5.5pt}{5.5pt}\selectfont}
\newcommand{\bahao}{\fontsize{5pt}{5pt}\selectfont}

% Heading style definitions
\makeatletter
\@addtoreset{section}{part} % Reset section counter with each part
\makeatother

% (一)(二)(三)等
\titleformat{\part}[block]{\kaishu\sihao \color[rgb]{0,0.439,0.753}}{\hspace*{2em}\textbf{(\chinese{part})}}{0.1em}{} 
\titlespacing*{\part}{0pt}{0pt}{0pt} % 缩进2字符,段前距0,段后距2pt
% 说明:Word模板定义,行距为固定行距22,稿纸设置中文本基线距离为15.6pt,正文字体为小四(12pt),蓝色标题的段后间距为0.5倍行距,大概为(22-15.6)/2 = 3pt,微调为2pt个人看起更接近于原Word模板。
% 不重要,WPS和Word打开文档格式都不一定一样,强迫症发疯设定😤。

% 1. 2. 3. 等
% Define the two section formats
\newcommand{\sectionplain}{%
    \titleformat{\section}[block]{\heiti \sihao}{\hspace*{2em}\heiti \thesection.}{0.5em}{}
}

\newcommand{\sectionblue}{%
    \titleformat{\section}[runin]{\kaiti \sihao \color[rgb]{0,0.439,0.753}}{\hspace*{2em}\heiti \thesection.}{0.5em}{}
}

\titlespacing*{\section}{0pt}{0pt}{0pt}
% Set default style
\sectionplain


% 1.1 1.2 1.3 等
\titleformat{\subsection}[block]{\heiti\sihao }{\hspace*{2em}\thesubsection}{0.5em}{}
\titlespacing*{\subsection}{0pt}{0pt}{0pt}

% 1.1.1 1.1.2 1.1.3 等
\titleformat{\subsubsection}[block]{\heiti\sihao }{\hspace*{2em}\thesubsubsection}{0.5em}{}
\titlespacing*{\subsubsection}{0pt}{0pt}{0pt}

% 参考文献标题样式
% Define bibliography heading with explicit formatting
\defbibheading{bibliography}[\refname]{%
    \section*{\heiti \sihao #1}  % Match your plain section format
    \markboth{#1}{#1}
}

% Set bibliography entries to \xiaosi
\renewcommand*{\bibfont}{\wuhao}

% Regular paragraph style
\setlength{\parindent}{2em} % 2em indent

% For algorithms
\renewcommand{\algorithmicrequire}{\textbf{输入:}}
\renewcommand{\algorithmicensure}{\textbf{输出:}}
\algnewcommand{\algorithmicnoind}[1]{\textbf{#1}}
\algnewcommand{\Noind}[1]{\item[\algorithmicnoind{#1}]}
\algrenewcommand{\algorithmiccomment}[1]{\hspace{1em} // #1}
\makeatletter
\renewcommand{\ALG@name}{算法}
\makeatother